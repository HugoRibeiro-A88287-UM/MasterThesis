%Statement of the problem?
%Why the problem appears?
%how solve the problem (design and methods)
%new contriubution



%################################################################
%why you did the work and what you were trying to achieve;

%What methods you used and what results you obtained?

%Conclusions

In recent years, the \glspl{mpsoc} complexity has been growing exponentially nevertheless, the performance of simulation tools is not 
following this growth, mainly because of their sequential simulation type. Therefore, simulation time increases each time a new \gls{mpsoc} 
is developed. Concerning this problem, the \gls{ice} \gls{rwth} Aachen team developed a parallel version of the atomic mode of Gem5, par-gem5. 
It is based on a synchronous \gls{pdes} which allows each simulation thread to run independently from the rest of the system for a time 
$t_{\Delta q}$ - the so-called quantum. Although this is a huge improvement, it carries a challenge in the quantum definition. 

This dissertation aims to solve the problem of finding the optimal quantum, which can lead to the best trade-off between accuracy and 
performance. Nowadays, par-gem5 is bound to a static quantum which cannot be changed during run-time however, an adaptive version can overcome 
this limitation. It was conceptualized and developed a flexible algorithm that can operate independently of the used benchmark and system.
Further, it was also tested and verified its functionality in a co-simulation environment.

In the end, the dynamic version of the quantum choice brings greater benefits when compared with the present. It was possible to achieve, 
on average, a performance gain of almost 10\%, only sacrificing 0.5\% of accuracy. Nevertheless, when a-priori information about the 
test is given, the tradeoff can be improved with the static approach. It becomes possible to calculate the optimal quantum before 
initiating the simulation, obtaining higher performances. If perfect accuracy is a requirement, the sequential version must be used, 
since the usage of both methods implies a loss in accuracy. Also, simulations with a single simulated core should always be 
executed with the later mode.


\paragraph{}\textbf{Keywords: Parallel Discrete Event Simulation, Gem5, Full-System Simulation, Quantum}
