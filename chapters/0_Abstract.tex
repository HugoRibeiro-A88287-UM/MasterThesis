%Statement of the problem?
%Why the problem appears?
%how solve the problem (design and methods)
%new contriubution



%################################################################
%why you did the work and what you were trying to achieve;

%What methods you used and what results you obtained?

%Conclusions

In recent years, the \glspl{mpsoc} complexity has been growing exponentially nevertheless, the performance of simulation tools is not 
following this growth, mainly because of their sequential simulation type. Therefore, simulation time increases each time a new \gls{mpsoc} 
is developed. Concerning this problem, the \gls{ice} \gls{rwth} Aachen team developed a parallel version of the atomic mode of Gem5, par-gem5. 
It is based on a synchronous \gls{pdes} which allows each simulation thread to run independently from the rest of the system for a time 
$t_{\Delta q}$ - the so-called quantum. Although this is a huge improvement, it carries a challenge in the quantum definition. 

This dissertation aims to solve the problem of finding the optimal quantum, which can lead to the best trade-off between accuracy and 
performance. Nowadays, par-gem5 is bound to a static quantum which cannot be changed during run-time however, an adaptive version can overcome 
this limitation. It was conceptualized and developed a flexible algorithm that can operate independently of the used benchmark and system.
Further, it was also tested and verified its functionality in a co-simulation environment. In the end, (...)


\paragraph{}\textbf{Keywords: Parallel Discrete Event Simulation, Gem5, Full-
System Simulation, Quantum}
\begin{comment}
\glsreset{FPGA}
\end{comment}

% taking into account the existing solutions.