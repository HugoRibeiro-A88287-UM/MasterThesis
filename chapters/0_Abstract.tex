%Statement of the problem?
%Why the problem appears?
%how solve the problem (design and methods)
%new contriubution



%################################################################
%why you did the work and what you were trying to achieve;

%What methods you used and what results you obtained?

%Conclusions

In the recent years, the \glspl{mpsoc} complexity has been growing exponentially, however the performance of simulation tools are not following this growth, mainly because of their sequential simulation type. Therefore, simulation time increases each time a new \gls{mpsoc} is developed. The \gls{ice} \gls{rwth} Aachen team developed a parallel version of the atomic mode of gem5. It’s based on a synchronous \gls{pdes} which allows each simulation thread to run independently from the rest of the system for a time $t_{\Delta q}$ – the so-called quantum.

This dissertation aims to solve the problem of finding the optimal quantum, which can lead to the best trade-off between accuracy and performance. Nowadays, gem5 is bound to a static quantum which cannot be changed during run-time, although an adaptive quantum can overcome the limitations of the static version. Thus, the main objective was to develop a flexible algorithm that can operate independently the used benchmark.  


\paragraph{}\textbf{Keywords: Parallel Discrete Event Simulation, Gem5, Full-
System Simulation, Quantum}
\begin{comment}
\glsreset{FPGA}
\end{comment}

% taking into account the existing solutions.