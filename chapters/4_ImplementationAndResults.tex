This chapter presents the chosen case study, which serves as a stimulus for system design and demonstrates 
the practical application of the developed work. The chosen case study focuses on the verification of 
data integrity, particularly concerning the \gls{crc} peripheral.

The \gls{crc} was created in 1961 by William Wesley Peterson \cite{peterson1961cyclic}. As the name suggests, 
it utilizes systematic cyclic codes to encode messages by incorporating a fixed-length check value. In the end, his work
contributed significantly to simplifying and enhancing the detection of accidental errors/changes in communication 
networks. \gls{crc} uses a generator polynomial, which is known by the sender and receiver, and it is used to 
perform the calculation. There are different standards however, the most common ones are the CRC-8, CRC-12, CRC-16, 
CRC-32, and CRC-CCIT \cite{borrelli2001ieee}

Another application for the \gls{crc} is the storage integrity. Due to defective components or electromagnetic fields,
bits can change their value without notice. In the presented case study, this scenario will be explored, where 
the \gls{crc} peripheral is used to maintain a specific memory state and verify if there have been any changes. 
Furthermore, to showcase the advantages of the previously developed work, the application will perform other 
tasks simultaneously. To implement the presented peripheral, the reference manual of the STM32F75xxx and 
STM32F74xxx microcontroller was taken into account.


\section{CRC peripheral} %Software development

\subsection{CRC in SystemC}

%Falar sobre o design no systemC
%A função de cada cena - initiator, route e crc

\subsection{CRC in gem5}

%CRC e considerado como um off chip peripheral
%Falar sobre de e necessario para assumir como um perifierico para a co-sim
%Necessario dizer ao gem5 que crc é um perifirico (armv7, adicionar ao realview)
%mostrar como foi implementado 

\subsection{TLM wrapper}

%Falar sobre este wrapper - O que é, porque existe(Porque o systemC e gem5 precisao de um "tradutor")
%Falar sobre a comunicação, do RP - feito com um socket
%Falar sobre a trama, playload e assim

\subsection{CRC application interface}

%Falar sobre o modulo CRC_SC
%Que funções tem, com aquele digrama de funcoes bonito
%Como existe com a biloteca HAL da STM


\section{Application simulation using gem5}

%Falar sobre as configurações da placa do vexpress
%Em que portas as coissas vao ficar connectadas
%perifericos que se vao utilizar

\subsection{CRC peripheral validation} 

%Show that the CRC pheripheral is working

\section{Memory integrity}

%Entrar aqui no tema do proposito do CRC

\subsection{Fault Modeling}

%Mostrar como é que vai ser injetada a falha na memoria

\section{Results}

%Mostrar os resultados da simulação com a falha

