%General Introduction about the thesis

%A normal is divided into five stages: define, ideate, prototype, production and deliver. One characteristic of this 
% process is its non-linearity. There is an interaction between the ideate, prototype, and production stages, the so-called interaction design. This 
% happens because problems or upgrades can be found, leading to a step back in the design process \cite{ProductDesignSteps}.


Designing an embedded system typically adopts a development model that matches the project requirements. 
For example, in the automotive industry, the most commonly adopted development model is the V-model \cite{liu2016incremental}. This model 
allows step-backs in the development process due to potential problems or upgrades \cite{ProductDesignSteps}. 

One example of this non-linear characteristic is in the development of an \gls{asic}. As the name suggests, an \gls{asic} is used for specific 
applications where dedicated hardware is required, for example, a critical system of a car. After the design, the prototype is developed, on 
which will be running tests, or benchmarks, in order to understand if the developed \gls{asic} satisfies all the requirements. Only in this stage, 
it is possible to obtain some indicators such as power consumption or compute performance to evaluate if \gls{swapc} requirements match. 
With \glspl{vp} or \glspl{fss}, on the other hand, most of the time is possible to have those without the physical prototype. 
Therefore, the time to market can be accelerated because problems or upgrades can be spotted much sooner.

Although these simulation tools are useful for the design of modern massively parallel and complex multi-core systems, 
many cannot execute a parallel simulation, in other words, they cannot parallelize simulation workload. 
As the complexity of the new systems increases, due to the integration of more and more applications on a single chip 
\cite{terascaleComputing}, simulation time increases.
For example, in the case of the SPEC2017 integer benchmark, where it may 
take up to two years to complete the simulation \cite{pargem5}. 

Another challenge in simulation is the communication between different tools. The system's complexity typically implies a simulation in 
different domains, since the integration of physical, software and network aspects is required \cite{gomes2017co}. Co-simulation is a technique 
used to model complex systems, allowing different stimuli from other simulators. The problem is that different tools can have different levels 
of abstraction, different communication protocols, etc. making it difficult to execute this technique directly. 

\section{Motivation}

Gem5 is an example of an \gls{fss} that cannot parallelize target workloads during simulation. To solve this, the \gls{ice} \gls{rwth} Aachen team 
developed par-gem5 \cite{pargem5}, a parallel version of the atomic mode of Gem5, that exploits the multithreading capabilities of modern host 
systems to parallelize the simulation itself. 
It is based on a synchronous \glsentryfull{pdes} where synchronizations are done periodically, according to a fixed-size time window called quantum 
or quanta.

High quantum allows for high simulation speeds but negatively impacts the simulation's accuracy, or, in the worst case, can even break the system's 
functionality. If the quantum is too small, the accuracy is perfect, although the simulation performance will be unsatisfactory. Thus, there is a 
tradeoff between accuracy and performance, and finding an optimal quantum is one of the main challenges when running synchronous \gls{pdes}.

In the current state of par-gem5 (and as in other simulators), the quantum is set once and then kept for the rest of the simulation. This brings 
several different problems. First of all, to know which is the best quantum, it is necessary to do the simulation in order to obtain the simulation 
results, and further evaluate if it was the best choice or not. Moreover, the quantum varies on different simulations, therefore there is not a 
single optimal quantum value that fits all simulations. The trial and error process, although feasible, consumes a lot of time, allowing for a 
study regarding the optimization of this optimal quantum-finding process. 

Another issue present in Gem5 is the lack of frameworks that allow co-simulation. Due to its open-source characteristics, it is possible to 
create or integrate any co-simulation environment. The problem is the time spent on this operation, in a way that is desirable to use other 
tools. Gathering this to the few available options at the present date, the Gem5 does not become an option.

\section{Goals and Contributions}

In the context of par-gem5, a dynamic quantum could address the previously mentioned issue by adjusting the quantum value for each simulation, 
leading to improved results. 
This approach is particularly advantageous because it compromises multiple phases with distinct computing and synchronization 
characteristics. Specifically, for the computational phase, increasing the quantum is beneficial, while for the synchronization phase, 
reducing the quantum is recommended.

With the dynamic quantum, it will be possible to automatically tune to the best value without any user inputs or feedback. 
Therefore, the quantum adaptation must be "on-the-fly” and be independent of the simulated system or benchmark. 
Alongside, upon finishing simulation runs, the simulator, at the end of the benchmark execution, should give feedback to the user, 
by the creation of a document including information related to the adaptive quantum.

In the context of co-simulation, a framework will be developed to facilitate the integration of different tools. Its design 
must be flexible, so it can be used with different simulators, and respect three characteristics in order to have a well-functioning
environment: data integrity, data exchange, and tool synchronization. Additionally, it should be a lightweight framework, 
ensuring that simulation overhead is not overly significant.

In the end, this dissertation development will not only aim to provide an algorithm that solves the problem of 
finding the best compromise between performance and accuracy, making it possible to simulate 
massively parallel and complex systems faster without a break in the accuracy, but also a co-simulation framework that will enable the use 
of Gem5 in projects with more application domains.

From an industry point of view, this work will grant a faster development of new products, in such a way companies can be the market leaders 
of the technology. Time-to-market can be optimized since the product is finished earlier, giving room for maneuvering to commercialize it at the 
right moment, and increasing the revenues.
 
\section{Dissertation Outline}

This document provides the development of a simulation extension that allows par-gem5 to automatically address the best quantum for the desired 
benchmark, together with a co-simulation framework proposal. Its content was divided into six chapters, which are going to be briefly described.

The second chapter introduces the concepts and methodologies used. The reader will receive notions about simulation in a general view, which 
will be specified in the embedded systems context. Topics such as simulation modes, simulation methodologies, and simulation tools 
will be covered, with special attention to Gem5, the simulator focused on this dissertation. \gls{ml} will be exploited, having in 
consideration \glspl{nn} and the simulation context. Finally, co-simulation practices are discussed, emphasizing their significance in the 
simulation coding.

The third chapter presents the developed work done regarding the dynamic quantum. It starts by giving a short explanation of the used benchmarks
to test the algorithms and required changes in par-gem5. Then, the four developed algorithms will be covered, providing an insight into the need to develop 
such algorithms and how they actually work. It ends with the final version, where a direct comparison with the static mode is made. The 
simulation results will be provided throughout the chapter in order to help the reader have a better idea of the algorithm's evolution.

The fourth chapter contains a proposal for a co-simulation environment, providing an interface where the two tools can cooperate. 
A case study was chosen the validate the developed work, giving a practical example of how this technique might be useful. 
It was based on a \gls{crc} peripheral, which is used for communication error detection and memory integrity checks. 
Since in a real-world scenario, there are other devices to manage, the framework was tested alongside other tasks. In the end, the integration
of the par-gem5 with the dynamic quantum was done, in order to evaluate its advantages and disadvantages in this situation. 

The final chapter discusses the conclusions regarding the developed work. It also includes future work, where can be found a set of 
aspects on which this work can have continuity and further improve par-gem5. 

