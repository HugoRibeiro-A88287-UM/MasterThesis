
\vspace*{-0.5cm}

Nos últimos anos, a complexidade dos sistemas com múltiplos processadores num \textit{chip} (\glspl{mpsoc}) tem crescido exponencialmente. 
Contudo, as ferramentas de simulação destes não têm seguido esse crescimento, principalmente devido ao seu tipo de simulação sequencial. 
Assim, os tempos de simulação aumentam sempre que um novo \gls{mpsoc} é desenvolvido. Nesse sentido, a equipa \gls{ice} da universidade 
\gls{rwth}, em Aachen, desenvolveu uma versão que possibilita a simulação paralela no modo atómico do Gem5, o Par-gem5. Esse modo é baseado 
numa simulação paralela síncrona de eventos discretos (\gls{pdes}) que permite cada \textit{thread} operar independentemente do 
resto do sistema por um tempo $t_{\Delta q}$, denominado quantum. Embora este trabalho tenha sido um importante avanço, ainda reside o problema 
de definir qual o melhor quantum para cada teste. Adicionalmente, à medida que os sistemas crescem em complexidade, a interação entre 
simuladores distintos aumenta devido à necessidade de avaliar diferentes aspetos. 

Esta dissertação tem como objetivo solucionar dois problemas. O primeiro consiste em encontrar o quantum ótimo, ou seja, o quantum que permita obter o 
melhor compromisso entre desempenho e a precisão de simulação. De momento, o Par-gem5 apenas permite a definição de um quantum 
estático, pelo que este não pode ser alterado durante o decorrer da simulação. Porém, um quantum dinâmico consegue ultrapassar essas limitações. 
Portanto, foi desenhado e desenvolvido um algoritmo flexível que consegue operar independentemente do teste e do sistema escolhido. O segundo 
problema reside na interação entre o Gem5 e outros simuladores. No seu estado atual, poucos conseguem trabalhar com esta ferramenta. Deste modo, foi 
proposto um ambiente de co-simulação que permite integrar qualquer simulador. Além disso, um caso de estudo para validar este ambiente foi escolhido. 

O trabalho desenvolvido resultou num algoritmo que trouxe grandes benefícios quando comparado com a solução atual. Foi possível atingir, 
em média, um aumento de desempenho de 10\%, sacrificando, apenas, 0.5\% de precisão. No entanto, quando existe informação prévia, 
este compromisso pode ser melhorado com a versão estática, pelo que esta não deve ser totalmente descartada. Caso seja necessária uma 
precisão perfeita, o modo sequencial deve de ser utilizado, já que os métodos anteriores implicam uma perda. Além disso, o ambiente 
de co-simulação proposto permitiu a interação com outros simuladores. Este manteve a integridade dos dados, a troca de dados, e a sincronização 
entre as ferramentas durante todo o processo. Com este trabalho, uma nova contribuição neste ramo foi desenvolvida.

\paragraph{}\textbf{Palavras-chave:} Simulação Paralela de Eventos Discretos, Gem5, Simulação de Sistemas Completos, Quantum, Co-Simulação
